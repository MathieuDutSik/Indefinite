\documentclass[12pt]{amsart}
\usepackage{amsfonts, amsmath, latexsym, epsfig}
\usepackage{amssymb}
\usepackage{epsf}
\usepackage{url}


\newcommand{\sfA}{\ensuremath{\mathsf{A}}}
\newcommand{\RR}{\ensuremath{\mathbb{R}}}
\newcommand{\NN}{\ensuremath{\mathbb{N}}}
\newcommand{\QQ}{\ensuremath{\mathbb{Q}}}
\newcommand{\CC}{\ensuremath{\mathbb{C}}}
\newcommand{\ZZ}{\ensuremath{\mathbb{Z}}}
\newcommand{\TT}{\ensuremath{\mathbb{T}}}
\newcommand{\R}{\ensuremath{\mathbb{R}}}
\newcommand{\N}{\ensuremath{\mathbb{N}}}
\newcommand{\Q}{\ensuremath{\mathbb{Q}}}
\newcommand{\C}{\ensuremath{\mathbb{C}}}
\newcommand{\Z}{\ensuremath{\mathbb{Z}}}
\newcommand{\T}{\ensuremath{\mathbb{T}}}
\newtheorem{proposition}{Proposition}
\newtheorem{theorem}{Theorem}
\newtheorem{corollary}{Corollary}
\newtheorem{algorithm}{Algorithm}
\newtheorem{lemma}{Lemma}
\newtheorem{problem}{Problem}
\newtheorem{conjecture}{Conjecture}
\newtheorem{claim}{Claim}
\newtheorem{remark}{Remark}
\newtheorem{definition}{Definition}
\def\QuotS#1#2{\leavevmode\kern-.0em\raise.2ex\hbox{$#1$}\kern-.1em/\kern-.1em\lower.25ex\hbox{$#2$}}


\urlstyle{sf}

\DeclareMathOperator{\Aut}{Aut}
\DeclareMathOperator{\Sym}{Sym}
\DeclareMathOperator{\Isom}{Isom}
\DeclareMathOperator{\vertt}{vert}
\DeclareMathOperator{\conv}{conv}
\DeclareMathOperator{\SC}{SC}
\DeclareMathOperator{\SL}{SL}
\DeclareMathOperator{\GL}{GL}
\DeclareMathOperator{\PSL}{PSL}
\DeclareMathOperator{\Out}{Out}
\DeclareMathOperator{\Min}{Min}
\DeclareMathOperator{\Dom}{Dom}
\DeclareMathOperator{\cone}{cone}
\DeclareMathOperator{\Stab}{Stab}


\begin{document}

\author{Mathieu Dutour Sikiri\'c}
\address{Mathieu Dutour Sikiri\'c, Rudjer Boskovi\'c Institute, Bijenicka 54, 10000 Zagreb, Croatia}
\email{mathieu.dutour@gmail.com}


\title{Manual of the GAP package {\tt indefinite}}
\date{}

\maketitle
\tableofcontents

\section{Installation}

A priori the system works only on unix/linux systems.
You need to follow the following steps:
\begin{enumerate}
\item The archive {\bf polyhedral.tar.gz} can be downloaded from
\url{http://mathieudutour.altervista.org/Polyhedral/index.html}

\item Previous to using polyhedral you need to install the GAP computer package (from \url{http://www.gap-system.org/}).

\item The archive {\bf polyhedral.tar.gz} should be untarred in the {\bf pkg} directory of GAP.

\item Your File {\bf .gap/gap.ini} must contain the following line:
\begin{verbatim}
SetUserPreference( "InfoPackageLoadingLevel", 4 ); # for additional debugging informations
SetUserPreference( "PackagesToLoad", [ "polyhedral"]);
\end{verbatim}
if you have no other needed packages. If the file is not existent then you need to create it.

\item Then one needs to run the {\bf configure} perl script in the {\bf polyhedral} directory in order to compile the external programs.
\end{enumerate}


\section{Introduction}
The package {\tt polyhedral} is designed to be used for doing all kinds
of computations related to polytopes and use their symmetry groups in
the course of the computation.
There are very many different functions but I will try here to explain
them as good as I can.

In order to install it, you should
\begin{enumerate}
\item Install {\tt GAP}.
\item Download {\tt indefinite.tar.gz} from 
{\url{http://www.liga.ens.fr/~dutour/Polyhedral}}
\item Unpack it in the directory {\tt pkg/} of the gap distribution.
\item Do {\tt ./configure} in order to compile the external programs.
\end{enumerate}




\subsection{Indefinite lattice functions}

{\bf General functions}
\begin{enumerate}
\item The function
\begin{verbatim}
INDEF_FORM_GetOrbitRepresentative:=function(Qmat, X)
\end{verbatim}
returns the orbit representative of vectors of norm $X$ in the lattice. if $X=0$ then the primitive isotropic vectors are returned.
\item The function
\begin{verbatim}
INDEF_FORM_StabilizerVector:=function(Qmat, v)
\end{verbatim}
returns the stabilizer of a vector in an indefinite lattice.
\item The function
\begin{verbatim}
INDEF_FORM_EquivalenceVector:=function(Qmat1, Qmat2, v1, v2)
\end{verbatim}
tests for equivalence of vectors v1 and v2 in their respective indefinite forms
\item The function
\begin{verbatim}
INDEF_FORM_AutomorphismGroup:=function(Qmat)
\end{verbatim}
computes the automorphism group of the indefinite quadratic form.
\item The function
\begin{verbatim}
INDEF_FORM_TestEquivalence:=function(Qmat1, Qmat2)
\end{verbatim}
test for equivalence of two indefinite quadratic forms.
\end{enumerate}





\end{document}
